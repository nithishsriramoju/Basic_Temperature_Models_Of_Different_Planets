\documentclass[a4paper,12pt]{article}
\usepackage{graphicx}
\usepackage{color}
\usepackage{amssymb}
\begin{document}
\title{{VERTICAL TEMPERATURE PROFILES OF PLANETS AND TITAN}\\\Large{PLANETARY ATMOSPHERES\\ASSIGNMENT 2}}
\date{}
\maketitle
\pagebreak
\begin{flushright}
\author{\Large S. NITHISH KUMAR\\}
SC17B159\\DEPARTMENT OF EARTH AND SPACE SCIENCES\\
INDIAN INSTITUTE OF SPACE SCIENCES AND TECHNOLOGY
\end{flushright}
\pagebreak
\section*{VENUS}
\includegraphics[scale=0.45]{D:/books and papers/sem 8/Planetary Atmospheres/Assignment_2/venus.png}
\subsection*{Discussion}
The vertical temperature profile of Venus is very close to modelled profile.

\section*{TITAN}
\includegraphics[scale=0.45]{D:/books and papers/sem 8/Planetary Atmospheres/Assignment_2/titan.png}
\subsection*{Discussion}
The vertical temperature profile of Titan is close upto 50 km. From 30 km the haze layer starts in Titan, which contains hydrocarbons, deviates temperature profile from modelled profile by absorbing incoming radiation. 


\section*{MARS}
\includegraphics[scale=0.45]{D:/books and papers/sem 8/Planetary Atmospheres/Assignment_2/mars.png}
\subsection*{Discussion}
The vertical temperature profile on mars is highly unstable due to frequent occurrence of dust storms. The profile is highly sensitive  to day-night cycle and yearly insolation cycle.Above picture is without dust intervention in temperature profile.The modelled profile's deviation is very high from observed profile.
 
\includegraphics[scale=0.45]{D:/books and papers/sem 8/Planetary Atmospheres/Assignment_2/mars_dust.png}
Above profile is plotted with dust intervention.The modelled profile's deviation is very high from observed profile. The slope is low compared to figure without dust intervention 




\end{document}
